\section*{توصیف اولیه}
\subsection*{مقدمه}
امروزه با توجه به کاربرد گسترده‌ی
\lr{Java}
و بالطبع
\lr{JVM}
،
در صنعت و جهان مدرن امروزی منطقی به نظر می‌رسد که فرآیند اجرای کدهای جاوا را سریع‌تر کنیم.
یکی از راه‌های خوب برای رسیدن به این مهم، می‌تواند پیاده‌سازی سخت‌افزاری 
\lr{JVM}
که در واقع هسته‌ی جاواست  باشد.
\subsection*{اهداف}
در این پروژه می‌خواهیم برای پردازنده‌ی 
\lr{ARM-7}(صبای ۲)
یک شتاب‌دهنده‌
\LTRfootnote{accelerator}
‌ی سخت‌افزاری
\lr{JVM}
بسازیم.
نحوه‌ی کار این شتاب‌دهنده به این شکل است که پردازنده 
\lr{opcode}های
\lr{JVM}
را دریافت می‌کند و به شتاب‌دهنده می‌دهد و شتاب‌دهنده دستورات معادل پردازنده را تولید می‌کند.
\subsection*{مراحل انجام پروژه}
به طور کلی با توجه به اهداف پروژه ما باید ۳ کار را برای انجام پروژه انجام‌دهیم:
\begin{itemize}
	\item 
	یادگیری کار با ماشین
	\lr{JVM}
	\item
	یادگیری کار با ماشین
	\lr{ARM-7}
	\item
	ساخت مبدل برای تبدیل دستورات میان این دو
\end{itemize}